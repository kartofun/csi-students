\documentclass[a4paper, 11pt]{article}
\usepackage[utf8]{inputenc}
\usepackage[english,russian]{babel}
\usepackage[T1, T2A]{fontenc}
\usepackage{graphicx}

\usepackage{pgfplots}
\usetikzlibrary{pgfplots.polar}
\pgfplotsset{compat=1.13, grid=major}
\usepackage[left = 2cm, right = 2cm, bottom = 2cm, top = 2cm]{geometry}
\usepackage[top=2cm, left=2cm, right=2cm, left=2cm]{geometry}
\usepackage{amsmath}

\usepackage{tabu}
\usepackage{threeparttablex} 
\usepackage{booktabs} 
\usepackage[tableposition=top]{caption}

\usepackage{subcaption}
\DeclareCaptionLabelFormat{gostfigure}{Рисунок #2}
\DeclareCaptionLabelFormat{gosttable}{Таблица #2}
\DeclareCaptionLabelSeparator{gost}{~---~}
\captionsetup{labelsep=gost}
\captionsetup[figure]{labelformat=gostfigure}
\captionsetup[table]{labelformat=gosttable}
\renewcommand{\thesubfigure}{\asbuk{subfigure}}
\captionsetup[table]{labelformat=simple, labelsep = endash, justification = raggedright, singlelinecheck = off}
\usepackage{indentfirst}
\graphicspath{{image/}}
\newcommand\tline[2]{$\underset{\text{#1}}{\text{\underline{\hspace{#2}}}}$}

% PGFPlots Table ========================================================
\usepackage{pgfplotstable}
\renewcommand{\arraystretch}{1.5}
% pgfplotstable settings
\pgfplotstableset{
    columns/w/.style = {column name = {\boldmath$\omega$}, column type = |c},
    columns/lg_w/.style = {column name = {\boldmath$\lg{\omega}$}, column type = |c},
    columns/A/.style = {column name = {\boldmath$A(\omega)$}, column type = |c},
    columns/L/.style = {column name = {\boldmath$20\lg{A(\omega)}$}, column type = |c},
    columns/psi/.style = {column name = {\boldmath$\psi$}, column type = |c|},
    every head row/.style = {before row = \hline},
    after row = {[1mm] \hline},
}

\begin{document}
	\begin{titlepage}
		\centering
		{\fontsize{12pt}{5cm}\selectfont \bfseries Министерство образования и науки Российской Федерации} \\ \vspace{0.5cm}
		{\fontsize{7pt}{5cm}\selectfont ФЕДЕРАЛЬНОЕ ГОСУДАРСТВЕННОЕ АВТОНОМНОЕ ОБРАЗОВАТЕЛЬНОЕ УЧРЕЖДЕНИЕ ВЫСШЕГО ПРОФЕССИОНАЛЬНОГО ОБРАЗОВАНИЯ} \\ 
		\vspace{1cm}
		{\fontsize{12pt}{5cm}\selectfont \bfseries САНКТ-ПЕТЕРБУРГСКИЙ УНИВЕРСИТЕТ ИНФОРМАЦИОННЫХ ТЕХНОЛОГИЙ, МЕХАНИКИ И ОПТИКИ} \\ \vspace{1.5cm}

		{\fontsize{14pt}{5cm}\selectfont Кафедра \hspace{1cm} \underline{Систем Управления и Информатики}  \hspace{1cm} Группа \underline{Р3340}} \\ 
		\vspace{2cm}

		{\fontsize{20pt}{5cm}\selectfont \bfseries Лабораторная работа №9} \\
		{\fontsize{20pt}{5cm}\selectfont \bfseries “Экспериментальное построение частотных характеристик типовых динамических звеньев”} \\
		{\fontsize{14pt}{5cm}\selectfont Вариант - 7} \\
		\vspace{1.5cm}

		\flushleft

		{Выполнил \hspace{2cm} \tline{(фамилия, и.о.)}{9cm} (подпись)} \\
		\vspace{2cm}

		{Проверил \hspace{2cm} \tline{(фамилия, и.о.)}{9cm} (подпись)} \\
		\vspace{5cm}

		"\underline{\hspace{0.7cm}}"\hspace{0.2cm}\underline{\hspace{2cm}}\hspace{0.2cm}20\underline{\hspace{0.7cm}}г. \hspace{2cm} Санкт-Петербург, \hspace{2cm} 20\underline{\hspace{0.7cm}}г. \\ \vspace{1cm}

		Работа выполнена с оценкой \hspace{1cm} \underline{\hspace{8cm}} \\ 
		\vspace{1cm}
		Дата защиты "\underline{\hspace{0.7cm}}"\hspace{0.2cm}\underline{\hspace{2cm}}\hspace{0.2cm}20\underline{\hspace{0.7cm}}г.

\end{titlepage}

\begin{center}
\section*{Задание}
\end{center}

\subsection*{Цель работы} 
\par
Изучение частотных характеристик типовых динамических звеньев и способов их построения; построение частотных характеристик, расчёт передаточных функций для заданных типовых звеньев.
\par
В работе предстоит построить АЧХ, ФЧХ, АФЧХ и ЛАФЧХ исследуемых звеньев, а также асимптотические ЛАЧХ, построенные графо-аналитическим методом. На вход исследуемого звена подаётся синусоидальный сигнал постоянной амплитуды. Надо измерить амплитуду выходного сигнала и сдвиг фаз между входным и выходным сигналами при различных частотах - таким образом будут получены данные для построения частотных характеристик. 

\begin{table}[h!]
    %\tabulinesep = 2mm

    \begin{threeparttable}
    	\caption{Исходные элементарные звенья}
    	\begin{tabular} {|l|c|}
        \hline
        	Тип звена & Передаточная функция \\ [0.5cm]  \hline
        	Интегрирующее с замедлением & $\displaystyle\frac{k}{s(Ts + 1)}$ \\ [0.5cm]  \hline
        	Изодромное & $\displaystyle\frac{k(Ts + 1)}{s}$ \\ [0.5cm]  \hline
        	Колебательное & $\displaystyle\frac{k}{T^2s^2 + 2{\xi}Ts + 1}$ \\ [0.5cm] \hline
    	\end{tabular}
    \end{threeparttable} 
\end{table}

\begin{table}[h!]
    \tabulinesep = 2mm
    	\caption{Параметры}\label{tab:perflogcross}
    	\begin{tabular}{|c|c|c|}
    		\hline
        	k & T & $\xi$ \\ \hline
        	3 & 5 & 0.4 \\
        	\hline
    	\end{tabular}
\end{table}

\newpage
\begin{center}
	\section{Исследование интегрирующего звена с замедлением}
\end{center}

\par 
Передаточная функция исследуемого звена:
\begin{align}
	W(s)=\frac{k}{s(Ts+1)}
\end{align}
\par 
Найдём выражения для АЧХ и ФЧХ:
\begin{align}
	W(j\omega)=\frac{-k(T\omega+j)}{\omega(T^2\omega^2+1)}
\end{align}

\begin{align}
	A(\omega)=\frac{k}{\omega\sqrt{T^2\omega^2+1}}
\end{align}
	
\begin{align}
	\psi(\omega)=arctg\frac{1}{T\omega}
\end{align}

\newpage
\par 
Данные, полученные по результатам моделирования, представлены в таблице 3.
\begin{table}[h!]
    \begin{threeparttable}
        \caption{Полученные данные} \label{tab:perflogcross}
        \pgfplotstabletypeset[]{data/integrate.txt}
    \end{threeparttable}
\end{table}

\newpage
\par 
На рисунке 1 представлены частотные характеристики интегрирующего звена с замедлением.

\begin{figure}[h!]
    \begin{subfigure}{0.5\textwidth}
        \centering
        \begin{tikzpicture}
            \begin{semilogxaxis} [
                    width = 0.9\textwidth,
                    xlabel = {$\omega$, 1/c},
                    ylabel = {$L_m$, дБ},
                    xmin = 10e-3,
                    xmax = 10e+1,
                   	extra y ticks = {23.52},
            		extra x ticks = {0.2},
            		xtick= {10e-3, 10e-1, 10e+0, 10e+1}
                ]
                \addplot table [x={w}, y={L}] {data/integrate.txt};
                \draw (0.01, 49.54) -- (0.2, 23.52) -- (100, -84.44);
                \draw[dashed] (0.01, 23.52) -- (0.2, 23.52);
                \draw[dashed] (0.2, 23.52) -- (0.2, -100);
            \end{semilogxaxis}
        \end{tikzpicture}
        \caption{График ЛАЧХ и асимптотическая ЛАЧХ}
    \end{subfigure}
    \begin{subfigure}{0.5\textwidth}
        \centering
        \begin{tikzpicture}
            \begin{semilogxaxis} [
                    width = 0.9\textwidth,
                    xlabel = {$\omega$, 1/c},
                    ylabel = {$\psi$, градусы},
                ]
                \addplot table [x={w}, y={psi}] {data/integrate.txt};
            \end{semilogxaxis}
        \end{tikzpicture}
        \caption{График ЛФЧХ}
    \end{subfigure}
    
    \vspace{0.5cm}

    \begin{subfigure}{0.5\textwidth}
        \centering
        \begin{tikzpicture}
            \begin{polaraxis} [
                    width = 0.9\textwidth,
                    xlabel = {$A(\omega)$},
                    ylabel = {$\psi$, градусы},
                ]
                \addplot table [x={psi}, y={A}] {data/integrate.txt};
            \end{polaraxis}
        \end{tikzpicture}
        \caption{График АФЧХ}
    \end{subfigure}
    \begin{subfigure}{0.5\textwidth}
        \centering
        \begin{tikzpicture}
            \begin{polaraxis} [
                    width = 0.9\textwidth,
                    xlabel = {$L_m$, дБ},
                    ylabel = {$\psi$, градусы},
                ]
                \addplot table [x={psi}, y={L}] {data/integrate.txt};
            \end{polaraxis}
        \end{tikzpicture}
        \caption{График ЛАФЧХ}
    \end{subfigure}
    \caption{Частотные характеристики интегрирующего звена с запаздыванием}
\end{figure}

\newpage
\begin{center}
	\section{Исследование изодромного звена}
\end{center}

\par 
Передаточная функция исследуемого звена:
\begin{align}
	W(s)=\frac{k(Ts + 1)}{s}
\end{align}
\par 
Найдём выражения для АЧХ и ФЧХ:
\begin{align}
	W(j\omega)=\frac{-k(T\omega+j)}{\omega(T^2\omega^2+1)}
\end{align}

\begin{align}
	A(\omega)=\frac{k\sqrt{T^2\omega^2+1}}{\omega}
\end{align}

\begin{align}
	\psi(\omega)=arctg\frac{1}{T\omega}
\end{align}

\par 
Данные, полученные по результатам моделирования, представлены в таблице 4.

\newpage
\begin{table}[h!]
    \begin{threeparttable}
        \caption{Полученные данные} \label{tab:perflogcross}
        \pgfplotstabletypeset[]{data/iso.txt}
    \end{threeparttable}
\end{table}

На рисунке 2 представлена временная диаграмма изодромного звена.
\begin{figure}
	\centering
	\begin{tikzpicture}
		\begin{axis}[
			width = 0.5\textwidth,
			xlabel = {t,c},
			ylabel = {y},
			legend pos = south west
			]
			\addplot [line width = 2, draw = blue] table [x={t}, y={y}] {data/test.txt};
			\addplot [line width = 2, dotted , draw = red] table [x={t},y={g}] {data/test.txt};
			
			\legend {
				$y(t)$,
				$g(t)$
				}
		\end{axis}
	\end{tikzpicture}
	\caption{Временная диаграмма изодромного звена при $\omega = 1$}
\end{figure}
\newpage
\par 
На рисунке 3 представлены частотные характеристики изодромного звена.

\begin{figure}[h!]
    \begin{subfigure}{0.5\textwidth}
        \centering
        \begin{tikzpicture}
            \begin{semilogxaxis} [
                    width = 0.9\textwidth,
                    xlabel = {$\omega$, 1/c},
                    ylabel = {$L_m$, дБ},
                    extra y ticks = { 23.52},
                    extra x ticks = {0.2},
                    xtick = {10e-3, 10e-1, 10e+0, 10e+1}
                ]
                \addplot table [x={w}, y={L}] {data/iso.txt};
                \draw (0.01, 50) -- (0.2, 23.52) -- (100, 23.52);
                \draw[dashed] (0.01, 23.52) -- (0.2, 23.52);
                \draw[dashed] (0.2, 23.52) -- (0.2, 0);
            \end{semilogxaxis}
        \end{tikzpicture}
        \caption{График ЛАЧХ и асимптотическая ЛАЧХ}
    \end{subfigure}
    \begin{subfigure}{0.5\textwidth}
        \centering
        \begin{tikzpicture}
            \begin{semilogxaxis} [
                    width = 0.9\textwidth,
                    xlabel = {$\omega$, 1/c},
                    ylabel = {$\psi$, градусы},
                ]
                \addplot table [x={w}, y={psi}] {data/iso.txt};
            \end{semilogxaxis}
        \end{tikzpicture}
        \caption{График ЛФЧХ}
    \end{subfigure}
    
    \vspace{0.5cm}

    \begin{subfigure}{0.5\textwidth}
        \centering
        \begin{tikzpicture}
            \begin{polaraxis} [
                    width = 0.9\textwidth,
                    xlabel = {$A(\omega)$},
                    ylabel = {$\psi$, градусы},
                ]
                \addplot table [x={psi}, y={A}] {data/iso.txt};
            \end{polaraxis}
        \end{tikzpicture}
        \caption{График АФЧХ}
    \end{subfigure}
    \begin{subfigure}{0.5\textwidth}
        \centering
        \begin{tikzpicture}
            \begin{polaraxis} [
                    width = 0.9\textwidth,
                    xlabel = {$L_m$, дБ},
                    ylabel = {$\psi$, градусы},
                ]
                \addplot table [x={psi}, y={L}] {data/iso.txt};
            \end{polaraxis}
        \end{tikzpicture}
        \caption{График ЛАФЧХ}
    \end{subfigure}
    \caption{Частотные характеристики изодромного звена}
\end{figure}

\newpage
\begin{center}
	\section{Исследование колебательного звена}
\end{center}

\par 
Передаточная функция исследуемого звена:
\begin{align}
	W(s)=\frac{k}{T^2s^2 + 2{\xi}Ts + 1}
\end{align}
\par 
Найдём выражения для АЧХ и ФЧХ:
\begin{align}
	W(j\omega)=\frac{k}{1-T^2\omega2+jT\xi\omega}
\end{align}

\begin{align}
	A(\omega)=\frac{k}{\sqrt{(1-T^2\omega^2)^2+(2T\xi\omega)^2}}
\end{align}

\begin{align}
	\psi(\omega)=-arctg\frac{2T\xi\omega}{1-T^2\omega^2}
\end{align}

\par 
Экспериментальные данные, полученные по результатам моделирования, представлены в таблице 5.
\newpage
\begin{table}[h!]
    \begin{threeparttable}
        \caption{Полученные данные} \label{tab:perflogcross}
        \pgfplotstabletypeset[]{data/koleb.txt}
    \end{threeparttable}
\end{table}

\newpage
\par 
На рисунке 4 представлены частотные характеристики колебательного звена.

\begin{figure}[h!]
    \begin{subfigure}{0.5\textwidth}
        \centering
        \begin{tikzpicture}
            \begin{semilogxaxis} [
                    width = 0.9\textwidth,
                    xlabel = {$\omega$, 1/c},
                    ylabel = {$L_m$, дБ},
                    extra y ticks = { 9.54},
                    extra x ticks = {0.2},
                    xtick = {10e-3, 10e-1, 10e+0, 10e+1}
                ]
                \addplot table [x={w}, y={L}] {data/koleb.txt};
                \draw (0.01, 9.54) -- (0.2, 9.54) -- (100, -98.41);
                \draw[dashed] (0.001, 9.54) -- (0.2, 9.54);
                \draw[dashed] (0.2, 9.54) -- (0.2, -110);
            \end{semilogxaxis}
        \end{tikzpicture}
        \caption{График ЛАЧХ и асимптотическая ЛАЧХ}
    \end{subfigure}
    \begin{subfigure}{0.5\textwidth}
        \centering
        \begin{tikzpicture}
            \begin{semilogxaxis} [
                    width = 0.9\textwidth,
                    xlabel = {$\omega$, 1/c},
                    ylabel = {$\psi$, градусы},
                ]
                \addplot table [x={w}, y={psi}] {data/koleb.txt};
            \end{semilogxaxis}
        \end{tikzpicture}
        \caption{График ЛФЧХ}
    \end{subfigure}
    
    \vspace{0.5cm}

    \begin{subfigure}{0.5\textwidth}
        \centering
        \begin{tikzpicture}
            \begin{polaraxis} [
                    width = 0.9\textwidth,
                    xlabel = {$A(\omega)$},
                    ylabel = {$\psi$, градусы},
                ]
                \addplot table [x={psi}, y={A}] {data/koleb.txt};
            \end{polaraxis}
        \end{tikzpicture}
        \caption{График АФЧХ}
    \end{subfigure}
    \begin{subfigure}{0.5\textwidth}
        \centering
        \begin{tikzpicture}
            \begin{polaraxis} [
                    width = 0.9\textwidth,
                    xlabel = {$L_m$, дБ},
                    ylabel = {$\psi$, градусы},
                ]
                \addplot table [x={psi}, y={L}] {data/koleb.txt};
            \end{polaraxis}
        \end{tikzpicture}
        \caption{График ЛАФЧХ}
    \end{subfigure}
    \caption{Частотные характеристики колебательного звена}
\end{figure}

\newpage
\begin{center}
	\section*{Вывод}
\end{center}
\par 
В лабораторной работе были исследованы следующие элементарные звенья: колебательное, интегральное с запаздыванием и изодромное. Были найдены частотные характеристики, а также построены графо-аналитическим методом асимптотические ЛАЧХ, к которым сходятся полученные с помощью математического моделирования графики. В области низких и высоких частот смоделированные ЛАЧХ асимптотически приближаются к прямым.	Для колебательного звена на частоте среза $\omega_c$ имеется "горб". Это объясняется явлением резонанса, а зависит высота такого пика от коэффициента затухания $\xi$, причём чем меньше значение $\xi$, тем больше высота. При $\xi=0$ звено вырождается в консервативное, а график на частоте среза претерпевает разрыв.
\par 
Согласно критерию Найквиста по полученным графикам можно определить устойчивость заданного звена. По АФЧХ исследуемых звеньев видно, что все звенья устойчивы, однако колебательное звено, что логично, находится на границе устойчивости при высоких частотах($\phi\approx-180$). Интегрирующее с замедлением звено также на высоких частотах почти выходит за границу устойчивости, что объясняется тем самым замедлением, которое существенно ухудшает характеристики системы, в частных случаях приводя её в состояние неустойчивости.
\end{document}
